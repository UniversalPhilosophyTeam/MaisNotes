\documentclass[11pt]{article}
\usepackage{geometry}                
\geometry{letterpaper}                 
\usepackage[parfill]{parskip}        
\usepackage{graphicx}
\usepackage{amssymb}
\usepackage{amsmath}
\usepackage{epstopdf}
\usepackage{verbatim}
\usepackage{float}
\usepackage{enumerate}
\usepackage{hyperref}
\usepackage[utf8]{inputenc}
\usepackage[T1]{fontenc}
\DeclareGraphicsRule{.tif}{png}{.png}{`convert #1 `dirname #1`/`basename #1 .tif`.png}
\usepackage{color}
\usepackage{textcomp}
\definecolor{listinggray}{gray}{0.9}
\definecolor{lbcolor}{rgb}{1,1,1}

\begin{document}

\section*{Problems for Discussion 3, 09/25/13}
Compiled by Mai Le, some problems from Prof. Yagle, Prof. Fessler

\section{Using DTFT Properties}
% Yagle hmwk 5, #5

Let $x[n] = \{1,4,3,2,5,7,\underline{-45},7,5,2,3,4,1\}$. Let $X(\omega)=DTFT(x[n])$. Compute the following without actually computing $X(\omega)$. Use DTFT properties!

\subsection{(a) $X(\pi)$}

{\color{blue}
$X(\pi)=\sum\limits_{n=-\infty}^\infty x[n]e^{-j\pi n} = \sum\limits_{n=-\infty}^\infty x[n](-1)^n$

$ = 1-4+3-2+5-7+-45-7+5-2+3-4+1 = -53$.
}

\subsection{(b) arg$X(\omega)$}

{\color{blue}
$x[n] = x[-n]$ so $X(\omega) = X^*(\omega)$. This implies that $X(\omega)$ is real.

$Re\{X(\omega)\} = Re\{\sum_{n=-\infty}^\infty x[n] e^{j \omega n} \} = \sum_{n=-\infty}^\infty Re\{x[n] e^{j \omega n} \}$

Each term $ Re\{x[n] e^{j \omega n} \} = Re\{x[n] cos(\omega n) + x[n] i\cdot sin(\omega n) \} = x[n] cos (\omega n) \leq x[n] $.

So, $Re\{X(\omega)\} = X(\omega) \leq \sum_{n=-\infty}^\infty x[n] = -45+2(1+4+3+2+5+7) = -1 < 0$

Therefore, $argX(\omega) = \pi$ for all $\omega$.

}

\subsection{(c) $\int_{-\pi}^\pi X(\omega) d\omega$}

{\color{blue}
We will use the following theorem, found on line 9 of table 2.2 (Fourier Transform Theorems):

$\sum\limits_{n=-\infty}^\infty x[n]y^*[n] = \frac{1}{2}\int_{-\pi}^\pi X(\omega)Y^*(\omega)d\omega $

$\int_{-\pi}^\pi X(\omega) d\omega = \int_{-\pi}^\pi X(\omega) 1(\omega) d\omega = \int_{-\pi}^\pi X(\omega) 1^*(\omega) d\omega$.

We know that $1(\omega)=1^*(\omega)$ has the IDTFT $\delta[n]$, so $\int_{-\pi}^\pi X(\omega) d\omega = 2\pi \sum\limits_{n=-\infty}^\infty x[n]\delta^*[n] = 2\pi \sum\limits_{n=-\infty}^\infty x[n]\delta[n]$

$=2 \pi x[0] = -90 \pi$

%This solution will be posted at a later date.
}


\subsection{(d) $\int_{-\pi}^\pi |X(\omega)|^2 d\omega$}

{\color{blue}

Use Parseval's Theorem!

Parseval's: $\sum\limits_{n=-\infty}^\infty |x[n]|^2 = \frac{1}{2\pi} \int_{-\pi}^\pi |X(\omega)|^2 d\omega$.

So, $\int_{-\pi}^\pi |X(\omega)|^2 d\omega = 2\pi \sum\limits_{n=-\infty}^\infty |x[n]|^2$


$ = 2\pi  \left(1^2+4^2+3^2+2^2+5^2+7^2 +(-45)^2+7^2+5^2+2^2+3^2+4^2+1^2 \right) = 4466 \pi$

}

\section{Computing the Z-Transform}
Compute the z-transform and the corresponding ROCs of the following sequences.

\subsection*{(a) $x[n]=\left(\frac{1}{5}\right)^nu[n]$}
% Yagle disc 4, #1

{\color{blue}
From $a^nu[n] \rightarrow \frac{z}{z-a}$, we get $(1/5)^nu[n] \rightarrow \frac{z}{z-1/5}$.

ROC: $|z| > |a| = 1/5$
}

\subsection*{(b) $x[n]=2^nu[-n]+\left(\frac{1}{3}\right)^nu[n]$}
% Yagle disc 4, #1

{\color{blue}
From $-a^nu[-n-1] \rightarrow \frac{z}{z-a}$, and manipulating $2^nu[-n]=2\cdot 2^{n-1}u[-(n-1)-1]$, we get $2^nu[-n] \rightarrow -\frac{2z}{z-2}z^{-1}=-\frac{1}{z-2}$. ROC: $|z| < |a| = 2$.

From $a^nu[n] \rightarrow \frac{z}{z-a}$, we get $(1/3)^nu[n] \rightarrow \frac{z}{z-1/3}$. ROC: $|z| > |a| = 1/3$

$X(z) = \frac{z}{z-1/3}-\frac{1}{z-2}$.

ROC: $1/3 < |z| < 2$.
}

\subsection*{(c) $x[n]=\{-1,\underline{0},1\}$}
% Yagle disc 4, #1

{\color{blue}
$X(z)=-1\cdot z+0\cdot 1 + 1\cdot z^{-1} = -z+z^{-1}$

ROC: $z \neq 0$
}

%\subsection*{$x[n]= \{1,3,\underline{4}\} $}
% Yagle hmwk 3, #1

%\subsection*{$x[n]= (2^n+1)u[n]$}
% Yagle hmwk 3, #1

%\subsection*{$x[n]= \left(\frac{1}{3}\right)^nu[n]+2^nu[-n-1]$}
% Yagle hmwk 3, #1

%\subsection*{$x[n]= 3^nu[n]+\left(\frac{1}{2}\right)^nu[-n-1]$}
% Yagle hmwk 3, #1

%\subsection*{$x[n]= (1+n)u[n]$}
% Fessler, hmwk 3, #2

%\subsection*{$x[n]= cos\left(\frac{\pi}{2}n-\frac{\pi}{3}\right)u[n-3]+\delta[n-2]u[n]$}
%% Fessler, hmwk 3, #2
%Don't use decimals for this one! Draw pole-zero plot.
%
%{\color{blue}
%
%}

\subsection*{(d) $x[n]= \sum_{k=-\infty}^n3^ku[n]$}
% Fessler, hmwk 3, #2

{\color{blue}
$x[n]= \sum_{k=-\infty}^n3^ku[n]$. 

Using the change of variables $l=n-k$, $x[n] = \sum\limits_{l=0}^\infty 3^{n-l}u[n]=3^nu[n]\sum\limits_{l=0}^\infty \left(\frac{1}{3}\right)^l$

Using the geometric series formula, $x[n] = 3^nu[n]\left(\frac{1}{1-\frac{1}{3}}\right) = \frac{3}{2}3^nu[n]$. 

Now we can apply the known z-transform pair to get $X(z) = \frac{3}{2} \frac{1}{1-3z^{-1}}$ with ROC $|z| > 3$.
}

\section{Z-Transform Manipulation}
Express the z-transform of $y[n]=\sum\limits_{k=-\infty}^nx[k]$ in terms of $X(z)$.

{\color{blue}
\begin{eqnarray*}
y[n] &=&x[n]+x[n-1]+x[n-2]+\cdots \\
Y(z) &=& X(z)+z^{-1}X(z)+z^{-2}X(z)+ \cdots  \text{ using the time shifting property of z-transform}\\
&=& X(z)\left(1+z^{-1}+z^{-2}+ \cdots\right) \\
&& \text{ use the geometric series formula ...} \\
&=& \frac{X(z)}{1-z^{-1}} \text{ if } |z| > 1 \\
\end{eqnarray*}

ROC: \{ROC of x\} $\cap \{|z| > 1\}$
}

\section{Inverse z-transform strategies}

Consider the z-transform $X(z) = \frac{1+2z^{-1}+z^{-2}}{1-\frac{3}{2}z^{-1}+\frac{1}{2}z^{-2}}$. We can rewrite $X(z)$ into a friendlier form $X(z) = B(z) + \frac{A_1}{D_1} +\frac{A_2}{D_2}$, where $B(z)$, $A_1(z)$, $A_2(z)$, $D_1(z)$, and $D_2(z)$ are polynomials of $z$ (or rather $z^{-1}$). 

Use long division to find $B(z)$ and partial fraction expansion to find the remaining $A_1(z)$, $A_2(z)$, $D_1(z)$, and $D_2(z)$.

{\color{blue}
To get $B(z)$, we divide $1+2z^{-1}+z^{-2}$ by $1-\frac{3}{2}z^{-1}+\frac{1}{2}z^{-2}$ until the polynomial order of the remainder is less than the polynomial order of the denominator (2 in this case). 

Doing this, we get $B=2$ with a remainder of $-1+5z^{-1}$, so $X(z) = 2+ \frac{-1+5z^{-1}}{1-\frac{3}{2}z^{-1}+\frac{1}{2}z^{-2}}$. 

Next we do partial fraction expansion of $\frac{-1+5z^{-1}}{1-\frac{3}{2}z^{-1}+\frac{1}{2}z^{-2}}$ to determine $A_1(z)$, $A_2(z)$, $D_1(z)$, and $D_2(z)$.

$\frac{-1+5z^{-1}}{1-\frac{3}{2}z^{-1}+\frac{1}{2}z^{-2}} = \frac{-1+5z^{-1}}{(1-\frac{1}{2}z^{-1})(1-z^{-1})}$, so $D_1(z) = 1-\frac{1}{2}z^{-1}$ and $D_2(z)=1-z^{-1}$.

Lastly, to find $A_1(z)$ and $A_2(z)$, we use equation 3.43:

$A_1(z)=(1-\frac{1}{2}z^{-1})\frac{-1+5z^{-1}}{(1-\frac{1}{2}z^{-1})(1-z^{-1})}\bigg|_{z=1/2} = -9$


$A_2(z)=(1-z^{-1})\frac{-1+5z^{-1}}{(1-\frac{1}{2}z^{-1})(1-z^{-1})}\bigg|_{z=1} = 8$

Finally, we have $X(z) = 2+ \frac{-9}{1-\frac{1}{2}z^{-1}}+\frac{8}{1-z^{-1}}$. This formulation is easier for finding the inverse z-transform because it is the sum of known z-transforms. Its ROC will be the intersection of the ROCs of each term.

Note: this example was covered in the textbook on pages 120-121 if you would like to read more details.
}

\end{document}