\documentclass[11pt]{article}
\usepackage{geometry}                
\geometry{letterpaper}                 
\usepackage[parfill]{parskip}        
\usepackage{graphicx}
\usepackage{amssymb}
\usepackage{amsmath}
\usepackage{epstopdf}
\usepackage{verbatim}
\usepackage{float}
\usepackage{enumerate}
\usepackage{hyperref}
\usepackage[utf8]{inputenc}
\usepackage[T1]{fontenc}
\DeclareGraphicsRule{.tif}{png}{.png}{`convert #1 `dirname #1`/`basename #1 .tif`.png}
\usepackage{color}
\usepackage{textcomp}
\definecolor{listinggray}{gray}{0.9}
\definecolor{lbcolor}{rgb}{1,1,1}

\begin{document}
{\small
\section*{Problems for Discussion 10, 12/04/13}
Compiled by Mai Le
}

\section{Some Types of Filters}

\begin{description}
\item[Notch] 
	\begin{itemize}
	\item Goal: remove particular frequencies. Frequency response of filter should have a "notch" (missing piece) at the undesired frequency.
	\item Place pairs of complex conjugate zeros on the unit circle
	\item Place poles near zeros to reduce the bandwidth of the notch
	\item Example: to eliminate the component at $0.4 \pi$
	\[ H(z) = \frac{(z-e^{j 0.4 \pi})(z-e^{-j0.4 \pi})}{(z-0.9e^{j 0.4 \pi})(z-0.9e^{-j0.4 \pi})} \]
	\end{itemize}
\item[Low pass] 
	\begin{itemize}
	\item Goal: Remove high frequencies.
	\item Place zeros near high frequencies
	\item Place poles near low frequencies
	\item Example:
	\[ H(z) = \frac{(z-e^{j0.5 \pi})(z-e^{-j0.5 \pi})(z-e^{j0.75 \pi})(z-e^{-j0.75 \pi})(z-e^{j \pi})}{(z-0.6)(z-0.8e^{j0.25 \pi})(z-0.8e^{-j0.25 \pi})(z-0.8e^{j0.5 \pi})(z-0.8e^{-j0.5 \pi})} \]
	\end{itemize}
\item[Comb] 
	\begin{itemize}
	\item Goal: add delayed version of signal to itself, used in feedforward and feedback systems.
	\item A notch filter in which the nulls occur periodically, like a comb
	\item Example:
	\vspace{3 cm}
	\end{itemize}
\end{description}

\section{Matlab example of filter anlaysis}

Let's use the LPF example:
	\[ H(z) = \frac{(z-e^{j0.5 \pi})(z-e^{-j0.5 \pi})(z-e^{j0.75 \pi})(z-e^{-j0.75 \pi})(z-e^{j \pi})}{(z-0.6)(z-0.8e^{j0.25 \pi})(z-0.8e^{-j0.25 \pi})(z-0.8e^{j0.5 \pi})(z-0.8e^{-j0.5 \pi})} \]

First we need to go from

\[
H(z) = \frac{b_0 z^{N-M}}{a_0} \frac{\prod_{k=1}^M (z-c_k)}{\prod_{k=1}^M (z-d_k)} \rightarrow H(z) = \frac{\sum_{k=0}^M b_k z^{-k}}{\sum_{k=0}^N a_k z^{-k}} 
\]

Then we can use the coefficients of the numerator and denominator, $b_k$ and $a_k$, respectively in \texttt{filter(b,a,x)} or to plot the transfer function with \texttt{freqz(b,a,n)}.

For rest of this problem, see Matlab script \texttt{disc10\_filter\_example.m}.

\section{Notch Filter Design}
Determine the coefficients of the following notch filter if 
\[ y[n] = a_1y[n-1] = x[n]+b_1x[n-1]+b_2x[n-2] \]

\begin{description}
\item[a)] It completely rejects the frequency component at $\omega = \frac{\pi}{4}$.
\item[b)] It amplifies a DC signal by 2.
\end{description}

\section{Effect of Noise on Deconvolution}

Suppose a klutzy cameraman moved the camera while snapping a photograph. From sensors, we happen to know the motion was to the right then downwards-right, corresponding to a convolution kernel that looks something like this:

\[ h[n,m] = \begin{bmatrix} 1 & 1 & 1 & 0 & 0\\
							0 & 0 & 0 & 1 & 0\\
							0 & 0 & 0 & 0 & 1 \end{bmatrix} \]
							
\subsection*{(a)} Let's use the idea of deconvolution to remove blur from the image. See \texttt{disc10\_deconv\_example.m}.

\subsection*{(a)} Suppose that the image was also corrupted by white Gaussian noise. How does this affect our deconvolution method? How does the SNR affect the performance of the algorithm?

\section{Separability of the DSFT and 2D-DFT}

As a reminder, here are the definitions of the DSFT (2D-DTFT) and the 2D-DFT:

\[ X(u,v) = \sum_{m=-\infty}^\infty \sum_{n=-\infty}^\infty x[n,m] e^{jum}e^{jvn} \]

\[ X[k,l] = X(u,v)\big|_{u = \frac{2 \pi k}{N}, v = \frac{2 \pi l}{M}}  =  \sum_{m=-\infty}^\infty \sum_{n=-\infty}^\infty x[n,m] e^{j\frac{2 \pi k}{N}m}e^{j\frac{2 \pi l}{M}n} \]

Note that we can have different N and M (dimensions of the 2D-DFT) but lecture had N=M. 

\begin{description}
\item[(a)] Show that the DSFT is equivalent to taking the DTFT in the $n$-direction, followed by a DTFT in the $m$-direction (or $m$ then $n$). Show that the 2D-DFT is equivalent to taking the DFT in the $n$-direction, followed by a DFT in the $m$-direction.
\item[(b)] Show that if $x[n,m]$ can be written as $x[n,m]=x_1[n]x_2[m]$, then $X(u,v)$ can be written as $X(u,v)=X_1(u)X_2(v)$.
\end{description}

\section{Practical Tip: Using \texttt{fftshift}}

What does \texttt{fftshift} do in 1D? What about 2D?

Can we reverse an \texttt{fftshift} with another \texttt{fftshift}? I.e. Does \texttt{fftshift(fftshift(a)) = a}?



\end{document}