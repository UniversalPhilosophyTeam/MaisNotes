\documentclass[11pt]{article}
\usepackage{geometry}                
\geometry{letterpaper}                 
\usepackage[parfill]{parskip}        
\usepackage{graphicx}
\usepackage{amssymb}
\usepackage{amsmath}
\usepackage{epstopdf}
\usepackage{verbatim}
\usepackage{float}
\usepackage{enumerate}
\usepackage{hyperref}
\usepackage[utf8]{inputenc}
\usepackage[T1]{fontenc}
\DeclareGraphicsRule{.tif}{png}{.png}{`convert #1 `dirname #1`/`basename #1 .tif`.png}
\usepackage{color}
\usepackage{textcomp}
\definecolor{listinggray}{gray}{0.9}
\definecolor{lbcolor}{rgb}{1,1,1}

\begin{document}

\section*{Problems for Discussion 3, 09/25/13}
Compiled by Mai Le, some problems from Prof. Yagle, Prof. Fessler

\section{Using DTFT Properties}
% Yagle hmwk 5, #5

Let $x[n] = \{1,4,3,2,5,7,\underline{-45},7,5,2,3,4,1\}$. Let $X(\omega)=DTFT(x[n])$. Compute the following without actually computing $X(\omega)$. Use DTFT properties!

\subsection{(a) $X(\pi)$}

%{\color{black}
%$X(\pi)=\sum\limits_{n=-\infty}^\infty x[n]e^{-j\pi n} = \sum\limits_{n=-\infty}^\infty x[n](-1)^n$
%
%$ = 1-4+3-2+5-7+-45-7+5-2+3-4+1 = -53$.
%}

\subsection{(b) arg$X(\omega)$}

%{\color{black}
%$x[n] = x[-n]$ so $X(w) = X^*
%}

\subsection{(c) $\int_{-\pi}^\pi X(\omega) d\omega$}

%{\color{black}
%$\int_{-\pi}^\pi X(\omega) d\omega = \int_{-\pi}^\pi \sum\limits_{n=-\infty}^\infty x[n]e^{-j\omega n} d\omega = \sum\limits_{n=-\infty}^\infty x[n]  \int_{-\pi}^\pi e^{-j\omega n} d\omega$
%
%$ = \sum\limits_{n=-\infty}^\infty x[n]  \left(\frac{1}{-j\omega n}e^{-j\omega n} \big|_{\omega=-\pi}^{\pi}  \right) = \sum\limits_{n=-\infty}^\infty x[n]  \left(\frac{-2}{j\omega n}(-1)^n \right) $
%
%IS THIS RIGHT
%}


\subsection{(d) $\int_{-\pi}^\pi |X(\omega)|^2 d\omega$}

%{\color{black}
%
%Use Parseval's Theorem!
%
%Parseval's: $\sum\limits_{n=-\infty}^\infty |x[n]|^2 = \frac{1}{2\pi} \int_{-\pi}^\pi |X(\omega)|^2 d\omega$.
%
%So, $\int_{-\pi}^\pi |X(\omega)|^2 d\omega = 2\pi \sum\limits_{n=-\infty}^\infty |x[n]|^2$
%
%
%$ = 2\pi  \left(1^2+4^2+3^2+2^2+5^2+7^2 +(-45)^2+7^2+5^2+2^2+3^2+4^2+1^2 \right) = 4466 \pi$
%
%}

\section{Computing the Z-Transform}
Compute the z-transform and the corresponding ROCs of the following sequences.

\subsection*{(a) $x[n]=\left(\frac{1}{5}\right)^nu[n]$}
% Yagle disc 4, #1

%{\color{black}
%From $a^nu[n] \rightarrow \frac{z}{z-a}$, we get $(1/5)^nu[n] \rightarrow \frac{z}{z-1/5}$.
%
%ROC: $|z| > |a| = 1/5$
%}

\subsection*{(b) $x[n]=2^nu[-n]+\left(\frac{1}{3}\right)^nu[n]$}
% Yagle disc 4, #1

%{\color{black}
%From $-a^nu[-n-1] \rightarrow \frac{z}{z-a}$, and manipulating $2^nu[-n]=2\cdot 2^{n-1}u[-(n-1)-1]$, we get $2^nu[-n] \rightarrow -\frac{2z}{z-2}z^{-1}=-\frac{1}{z-2}$. ROC: $|z| < |a| = 2$.
%
%From $a^nu[n] \rightarrow \frac{z}{z-a}$, we get $(1/3)^nu[n] \rightarrow \frac{z}{z-1/3}$. ROC: $|z| > |a| = 1/3$
%
%$X(z) = \frac{z}{z-1/3}-\frac{1}{z-2}$.
%
%ROC: $1/3 < |z| < 2$.
%}

\subsection*{(c) $x[n]=\{-1,\underline{0},1\}$}
% Yagle disc 4, #1

%{\color{black}
%$X(z)=-1\cdot z+0\cdot 1 + 1\cdot z^{-1} = -z+z^{-1}$
%
%ROC is all but origin
%}

%\subsection*{$x[n]= \{1,3,\underline{4}\} $}
% Yagle hmwk 3, #1

%\subsection*{$x[n]= (2^n+1)u[n]$}
% Yagle hmwk 3, #1

%\subsection*{$x[n]= \left(\frac{1}{3}\right)^nu[n]+2^nu[-n-1]$}
% Yagle hmwk 3, #1

%\subsection*{$x[n]= 3^nu[n]+\left(\frac{1}{2}\right)^nu[-n-1]$}
% Yagle hmwk 3, #1

%\subsection*{$x[n]= (1+n)u[n]$}
% Fessler, hmwk 3, #2

%\subsection*{$x[n]= cos\left(\frac{\pi}{2}n-\frac{\pi}{3}\right)u[n-3]+\delta[n-2]u[n]$}
%% Fessler, hmwk 3, #2
%Don't use decimals for this one! Draw pole-zero plot.
%
%{\color{black}
%
%}

\subsection*{(d) $x[n]= \sum_{k=-\infty}^n3^ku[n]$}
% Fessler, hmwk 3, #2

%{\color{black}
%
%}

\section{Z-Transform Manipulation}
Express the z-transform of $y[n]=\sum\limits_{k=-\infty}^nx[k]$ in terms of $X(z)$.

%{\color{black}
%\begin{eqnarray*}
%y[n] &=&x[n]+x[n-1]+\cdots \\
%Y(z) &=& X(z)+z^{-1}X(z)+ \cdots \\
%&=& \frac{X(z)}{1-z^{-1}} \text{ if } |z| > 1 \\
%\end{eqnarray*}
%
%ROC: ROC of x $\cap \{|z| > 1\}$
%}

\section{Inverse z-transform strategies}

Consider the z-transform $X(z) = \frac{1+2z^{-1}+z^{-2}}{1-\frac{3}{2}z^{-1}+\frac{1}{2}z^{-2}}$. We can rewrite $X(z)$ into a friendlier form $X(z) = B(z) + \frac{A_1}{D_1} +\frac{A_2}{D_2}$, where $B(z)$, $A_1(z)$, $A_2(z)$, $D_1(z)$, and $D_2(z)$ are polynomials of $z$ (or rather $z^{-1}$). 

Use long division to find $B(z)$ and partial fraction expansion to find the remaining $A_1(z)$, $A_2(z)$, $D_1(z)$, and $D_2(z)$.

\end{document}